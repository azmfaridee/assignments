\documentclass{beamer}

 \usepackage{beamerthemesplit}
% \usepackage[danish]{babel}
\usepackage[utf8]{inputenc}
% \usepackage{graphics}
\usepackage{graphicx}
% \usepackage{epsfig}
\usepackage{subfigure}
 \usepackage{url}
 \usepackage{amsmath}
\usepackage{amssymb}
 \usepackage{algorithmic}

% \DeclareGraphicsExtensions{.pdf,.png,.jpg,.eps}
\graphicspath{{./}}

\definecolor{kugreen}{RGB}{50,93,61}
\definecolor{kugreenlys}{RGB}{132,158,139}
\definecolor{kugreenlyslys}{RGB}{173,190,177}
\definecolor{kugreenlyslyslys}{RGB}{214,223,216}

\setbeamercovered{transparent} \mode<presentation> {
  \usetheme{Copenhagen} \usecolortheme[named=kugreen]{structure}
  \useinnertheme{circles} \usefonttheme[onlymath]{serif}
  \setbeamercovered{transparent}
  \setbeamertemplate{blocks}[rounded][shadow=true] }
% \setbeamertemplate{background}{\includegraphics[width=1\textwidth]{natfak_baggrund.pdf}}
% \logo{\includegraphics[width=1cm]{billeder/logo}}

\title{High Performance Query Execution in a Scalable Heterogeneous Distributed Environment with Fault Tolerance}
% \subtitle{}
\author{Abu Zaher Md. Faridee\\  Samir Hasan \\ Wasik Mursalin Rushafi}
\institute{Department of Computer Science\\Bangladesh University of
  Engineering and Technology} \date{\today{}}

\begin{document}

\frame{ \titlepage
  % \vspace{-0.5cm}
  % \begin{center}
  %   %   \includegraphics[height=0.25\textheight]{billeder/polygons}
  % \end{center}
}

\frame{ \frametitle{Overview}
  \tableofcontents[pausesection]
}

% \begin{block}{Kryptering med kaotiske kredsløb}
%   \begin{columns}
%     \column{.3\textwidth} \hspace{0.5cm}
%       %     \includegraphics[width=0.7\textwidth]{billeder/circuit}
%     \column{.7\textwidth} \textit{Mogens Høgh Jensen}, NBI
%   \end{columns}
% \end{block}

\section{Introduction}
\label{sec:introduction}

% \subsection{The Longest Path Problem}
% \label{sec:problem-definition}

\subsection{Problem Definition}
\label{sec:problem-definition}


% \frame{

%   \frametitle{The Longest Path Problem}

%   \begin{block}{Definition}

%     Given an undirected graph $G = (V, E)$, where $V$ is the set of
%     $n$ vertices and $E$ is the set of $m$ edges, for all $u, v \in
%     V$, find the highest cost path from $u$ to $v$, visiting a vertex
%     only once
%   \end{block}

%   \begin{itemize}
%   \item For \emph{Weighted Graphs}, the \emph{Longest Path} is the
%     path that has the highest sum of weights of it's edges
%   \item For \emph{Unweighted Graphs}, the \emph{Longest Path} is the
%     path that contains the most number of edges
%   \item Covering all the vertices is not mandatory
%   \end{itemize}

% }


% \frame{
%   \frametitle{Example}
%   The bold edges denote the longest path for this graph, total cost is
%   45
%   \begin{center}
%     \includegraphics[width=0.7\textwidth]{lpath-best-sol}
%   \end{center}
% }

% \frame{
%   \frametitle{NP-Completeness}
%   \begin{block}{Reduction from \emph{Hamiltonian Path} Problem}
%     \begin{itemize}
%     \item Given an instance $G' = (V' ,E')$ for \emph{Hamiltonian
%         Path}, count the number $|V'|$ of nodes in $G'$ and output the
%       instance $G = G',K = |V '|$ for \emph{Longest Path}.
%     \item $G'$ has a simple path of length $|V'|$ if and only if
%       $|G'|$ has a Hamiltonian path.
%     \end{itemize}
%   \end{block}
%   The problem is NP-Complete even if there is no \emph{Hamiltonian
%     Path}, as we are trying to visit as much vertices as possible }


\subsection{Motivation}
\label{sec:motivation}

% \frame{
%   \frametitle{Applications}
%   \begin{itemize}
%   \item Appears as a subproblem in path planning, networking and in
%     many other industrial and logistic applications
%   \item Portugal and Rocha [cite needed] used \emph{Longest Path} to
%     patrol a graph with multiple robots when there is no
%     \emph{Eulerian} and \emph{Hamiltonian Path}
%   \end{itemize}
% }

\subsection{Research Challenges}

% \frame{
%   \frametitle{Previous Works }

%   \begin{itemize}
%   \item Karger [cite needed] first proposed a polynomial time
%     approximation algorithm for weighted undirected graphs, but with
%     limited performance
%   \item Portugal and Rocha [cite needed] used a \emph{Genetic
%       Algorithm} based approach in 2010
%   \end{itemize}
% }


\section{State of the Art}
\label{sec:state-of-the-art}

% \frame{

%   \frametitle{Ant Colony Optimization}
%   \begin{itemize}
%   \item A nature inspired algorithm based on the foraging behavior of
%     ants
%     % \item It's a constructive metaheuristic
%   \item Has been successfully applied to solve several NP-hard
%     combinatorial optimization problems [cite needed], such as
%     traveling salesman problem [cite needed], vehicle routing problem
%     [cite needed], and quadratic assignment problem [cite needed]
%   \end{itemize}

% }

% \frame{
%   \frametitle{The Basics of ACO}
%   \begin{itemize}
%   \item Model the given problem as a searching problem in a graph
%   \item Use artificial ants to construct the best path
%   \item Each ant puts a `trail' on it's discovered solution called
%     `\emph{pheromone}'
%   \item Each ant uses this \emph{`pheromone'} information to gradually
%     construct a better solution
%   \item This \emph{`pheromone'} is the globally shared memory of the
%     ants and the most powerful feature of ACO
%   \end{itemize}
% }

% \frame{
%   \frametitle{The Basic Equations of ACO}
%   \begin{block}{The probabilistic Formula}
%     $p_{k}(r, s) = \left\{
%       \begin{array}{rcl}
%         \frac{\tau(r, s) * \eta(r, s)^{\beta}}{\sum_{u \in M_k}{\tau(r,
%             s) * \eta(r, s)^{\beta}}} & $ if $ s \in M_k\\
%         0 & $ otherwise $
%       \end{array}
%     \right.$
%   \end{block}
% }


\section{Advantages and Disadvantages}
\label{sec:advantages-and-disadvantages}

% \frame{

%   \frametitle{Our Ant Colony Based Approach on Longest Path}
%   \begin{block}{Input}
%     \begin{itemize}
%     \item Adjacency matrix representation of the input graph
%     \item Parameter for ACO: $\alpha$ and $\beta$
%     \end{itemize}
%   \end{block}

%   \begin{block}{Output}
%     \begin{itemize}
%     \item The \emph{Longest Path} (order of nodes to visit)
%     \item Cost of the \emph{Longest Path}
%     \end{itemize}
%   \end{block}

%   Most of the time $\alpha = 3$ and $\beta = 2$ produces the best
%   result [cite needed] }

% \subsection{The Heuristic Function}

% \frame{
%   \frametitle{The Heuristic Function}

%   \begin{block}{Value of the Heuristic Function $\eta(r, s)$}
%     \begin{itemize}
%     \item For weighted graphs, $\eta(r, s) = cost(r, s)$, where
%       $cost(r, s)$ is the weight/cost from node $r$ to node $s$
%     \item For un-weighted graphs, $\eta(r, s) = nbrcount(s)$ where
%       $nbrcount(s)$ is the number of neighbors of node $s$
%     \end{itemize}
%   \end{block}
% }

% \subsection{Local Search}

% \frame{
%   \frametitle{Local Search}
%   \begin{itemize}
%   \item It is a well known fact that basic ACO algorithm is prone to
%     \emph{Local optima}
%   \item Two types of \emph{Local Search} implemented to circumvent
%     this issue
%     \begin{itemize}
%     \item The Wise Ant
%     \item The Missing Vertex Checker
%     \end{itemize}
%   \item After all the ants have completed their tour, the local best
%     solution is put through these \emph{Local Search} procedures
%   \end{itemize}
% }

% \frame{\frametitle{Local Search: Procedure 1}
%   \begin{block}{Algorithm: The Wise Ant}
%     \begin{itemize}
%     \item Given a graph $G = (V, E)$ and a solution $S = [v_1, v_2,
%       v_3, \ldots ~ \ldots, v_m]$, pick two consecutive nodes $v_i,
%       v_{i+1}$ randomly from $S$ where $i \geq \lfloor m/2 \rfloor$
%     \item Reconstruct the solution with the added restriction that the
%       ant cannot go to $v_{i+1}$ from $v_i$
%     \end{itemize}
%   \end{block}
%   \begin{itemize}
%   \item If the local search produces a better solution, update the
%     local best solution with this one
%   \item To maximize efficiency, this search is applied to given
%     solution twice, in both forward and reverse order

%   \end{itemize}
% }

% \frame{
%   \frametitle{Example: The Wise Ant - Before Local Search}
%   Current cost is 43, current vertex is marked as \emph{grey}, tabu
%   vertex is marked as \emph{green}
%   \begin{center}
%     \includegraphics[width=0.7\textwidth]{lpath-before-wiseant}
%   \end{center}
% }

% \frame{
%   \frametitle{Example: The Wise Ant - After Local Search}
%   After local search cost is maximized to 45
%   \begin{center}
%     \includegraphics[width=0.7\textwidth]{lpath-best-sol}
%   \end{center}

% }

% \frame{
%   \frametitle{Local Search: Procedure 2}
%   Given a graph $G = (V, E)$, a solution $S = [v_1, v_2, v_3, \ldots ~
%   \ldots, v_m]$ and list of unvisited nodes $U = [u_1, u_2, u_3,
%   \ldots ~\ldots, u_{n-m}]$ % where $\forall{ v_i, u_j} \in V$,
%   do the following:
%   \begin{block}{Algorithm: The Missing Node Checker}
%     \begin{algorithmic}
%       \FORALL{$u_j \in U$} \FORALL{$v_i \in S$} \IF{$ cost(v_i, u_j) +
%         cost(u_j, v_{j+1}) \geq cost(v_i, v_{i+1})$} \STATE Insert
%       $u_j$ into $S$ after $v_i$
%       \ENDIF
%       \ENDFOR
%       \ENDFOR
%     \end{algorithmic}
%   \end{block}
% }

% \section{Experimental Results}
% \label{sec:experimental-results}

% \frame{\frametitle{Example: Missing Vertex Checker - Before Local
%     Search} Unvisited node is marked as \emph{green}, two of it's
%   adjacent visited nodes are marked as \emph{grey}, current cost is 38
%   \begin{center}
%     \includegraphics[width=0.7\textwidth]{lpath-missing-vertex-before}
%   \end{center}
% }

% \frame{\frametitle{Example: Missing Vertex Checker - After Local
%     Search} Missing vertex added to the path, now cost is 43
%   \begin{center}
%     \includegraphics[width=0.7\textwidth]{lpath-missing-vertex-after}
%   \end{center}
% }

% \frame{
%   \frametitle{Experimental Results}
%   \begin{itemize}
%   \item We compare our results with Portugal and Rocha's [cite needed]
%     \emph{Genetic Algorithm} (GA) based approach
%   \item They implemented 4 variants, two crossover based, one mutation
%     based and one variant have both mutation and crossover
%   \item Mutation based variant is the fastest one with good solution
%     quality, we compare ours results with this variant
%   \item We separately consider two type of graph classes,
%     \emph{sparse} and \emph{dense}
%   \end{itemize}
% }

% \subsection{Sparse Graphs}

% \frame{
%   \frametitle{Sparse Graphs}
%   \begin{itemize}
%   \item Input graph has 134 nodes and 134 edges with an optimum value
%     of 1556.0 for the longest path
%   \item The GA is run with 400 \emph{chromosomes} for 10 iterations
%   \item The ACO is run with 100 \emph{ants} for 10 iterations
%   \end{itemize}
%   \begin{block}{Comparison}
%     \begin{center}
%       \begin{tabular}[c]{c c c}
%         & Genetic Algorithm & ACO\\
%         \hline
%         Success Rate & 34.0\% & 90.0\%  \\
%         Quality &  95.7\% & 97.8\% \\
%         Avg. Running Time (ms) & 319.3 & 297.5 
%       \end{tabular}
%     \end{center}
%   \end{block}
%   Success rate is calculated by running the whole algorithm 100 times
% }

% \subsection{Dense Graphs}

% \frame{
%   \frametitle{Dense Graphs}
%   \begin{itemize}
%   \item Input graph has 30 nodes, 269 edges with optimum value 274.0
%     for longest path
%   \item The GA is run with 400 \emph{chromosomes} for 100 iterations
%   \item The ACO is run with 100 \emph{ants} for 100 iterations
%   \end{itemize}
%   \begin{block}{Comparison}
%     \begin{center}
%       \begin{tabular}[c]{c c c}
%         & Genetic Algorithm & ACO\\
%         \hline
%         Success Rate & 0\% & 20\%  \\
%         Quality &  89.6\% & 99.4\% \\
%         Avg. Running Time (ms) &  3546.7 & 2164.8 
%       \end{tabular}
%     \end{center}
%   \end{block}
% }

\section{Future Research}
\label{sec:future-research}


\section{Conclusion}
\label{sec:conclusion}

% \frame{
%   \frametitle{Conclusion}
%   \begin{itemize}
%   \item Preliminary results show that our ACO based implementation
%     performs way better than GA.
%   \end{itemize}
% }

\frame{
  \frametitle{Thank You}
  \begin{center}
    \large{Questions \& Answers}
  \end{center}
}

\end{document}
